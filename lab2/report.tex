\documentclass[10pt,a4paper]{article}
\usepackage[margin=1in]{geometry}
\usepackage{titling}
\usepackage{xeCJK}
\usepackage{indentfirst}
\usepackage{enumitem}
\setCJKmainfont{AR PL UMing TW}

\title{WNFA Lab 2 Report}
\author{Group 5: & B02902125吳哲宇 & B01902010李紹詮 & B01902048王欣維 \\
    & B01902085邱劭同 & B01902138蔡存哲}
\preauthor{\begin{center}\begin{tabular}{lccr}}
\postauthor{\end{tabular}\end{center}}
\date{}

\usepackage{fancyhdr}
\pagestyle{fancy}
\lhead{\bfseries\thetitle}
\chead{}
\rhead{Group 5}
\lfoot{}
\cfoot{}
\rfoot{\thepage}
\renewcommand{\headrulewidth}{0.4pt}
\renewcommand{\footrulewidth}{0.4pt}

\setlength{\parindent}{2em}
\setlength{\parskip}{.4em}

\begin{document}
\maketitle
\thispagestyle{fancy}

\section*{工作原理}
利用JJY/WWVB訊號讓電波鐘校正時間。JJY/WWVB time code 是有一定格式的時間表示法,每一秒鐘傳送一個訊號,利用duty cycle來區分0、1、Marker,而完整的時間資料需要用一分鐘來傳送。

產生訊號的部分,以電流產生磁場,產生對應訊號的原理,在Zigduino上以CTC mode產生40/60kHz的方波,並按照JJY/WWVB time code的規則產生一連串的電流,再將電線纏繞多圈以增強磁場強度,使電波鐘順利收到訊號並同步時間。
\section*{遇到的問題}
\begin{enumerate}
    \item 用固定的pattern測試時,電波鐘對時間訊號沒有反應。
    \item 輸入只有年月日,但JJY要求需要有星期欄位。
\end{enumerate}

\section*{解決方法}
\begin{enumerate}
    \item 電波鐘本身對錯誤的時間訊號不會有反應,一開始的code將時間固定,每次傳送出去的都是相同的時間,而電波鐘對於不會增加的時間訊號,會視為錯誤的訊號因此不會進行同步,而後來將傳送的時間改成像真實世界一樣,會不斷的增加後,電波鐘就成功的同步了。
    \item 手寫萬年曆。
\end{enumerate}

\section*{工作分配}
\begin{itemize}[leftmargin=!,itemindent=-4em]
    \item 吳哲宇:寫輸入處理和萬年曆計算
    \item 李紹詮:實作JJY,程式碼架構
    \item 王欣維:報告統整
    \item 邱劭同:實作WWVB
    \item 蔡存哲:意見領袖
\end{itemize}

\end{document}
