\documentclass[10pt,a4paper]{article}
\usepackage[margin=1in]{geometry}
\usepackage{titling}
\usepackage{xeCJK}
\usepackage{indentfirst}
\usepackage{enumitem}
\setCJKmainfont{AR PL UMing TW}

\title{WNFA Lab 1 Report}
\author{Group 5: & B02902125吳哲宇 & B01902010李紹詮 & B01902048王欣維 \\
    & B01902085邱劭同 & B01902138蔡存哲}
\preauthor{\begin{center}\begin{tabular}{lccr}}
\postauthor{\end{tabular}\end{center}}
\date{}

\usepackage{fancyhdr}
\pagestyle{fancy}
\lhead{\bfseries\thetitle}
\chead{}
\rhead{Group 5}
\lfoot{}
\cfoot{}
\rfoot{\thepage}
\renewcommand{\headrulewidth}{0.4pt}
\renewcommand{\footrulewidth}{0.4pt}

\setlength{\droptitle}{-1cm}
\setlength{\parindent}{2em}
\setlength{\parskip}{.4em}

\begin{document}
\maketitle
\thispagestyle{fancy}

\section*{工作原理}
利用綠光作定位條紋,以16條為一組,並交替顯示兩種pattern用以區分前後frame。而紅色和藍色用來傳輸資料,因此一個frame可以傳輸32bit。

解碼時先看green channel,使用OpenCV中的canny edge detect找出邊界,並且判斷綠光的pattern,分別為"0101010101010111"和"0101010101110111"兩種,這時就可以用定位條紋找出紅藍光代表的sequence。

由於有時綠色LED旁的藍色LED太強會干擾綠光影像,因此在分析green channel定位條紋時,會將圖片切成左右兩半,都解碼定位條紋,看哪邊能解出合法的pattern。做邊緣偵測找定位條紋時,會有很多雜訊,導致在條紋邊緣會找到大量的分界線。因此會用影像中最寬的條紋寬度的八分之一當threshold,合併過於靠近的分界線,得到真正條紋的分界線。

當判斷單一條紋代表的bit時,先將整個條紋的值(0\textasciitilde255)作平均,用這個平均值的三分之二當作threshold,當條紋中央2/3的平均值大於threshold時為1,反之為0。

為了減少傳輸資料時的閃爍跟色彩改變,有簡易的將資料XOR上一組亂數,雖然可減輕閃爍跟色彩變化,但還是人眼還是能區分出差異,效果沒有十分良好。

每frame共有紅、藍各16bit,合計32bit;使用的相機為30fps,data rate粗估約960 bit/s。

\section*{遇到的問題}
\begin{enumerate}
    \item 手機內建的相機無法精確的控制。
    \item 我們使用了OpenCV的邊緣偵測,由於計算量相當大,可以預期在手機上跑會很慢,加上組內人對Android不熟,沒時間研究如何把OpenCV跑在手機上。
    \item 光源太強時會溢到其他條紋,造成解碼困難。
\end{enumerate}

\section*{解決方法}
\begin{enumerate}
    \item 使用Android 5.0提供的camera2 API寫一個簡單的錄影程式,可以自己調整光圈、快門等。
    \item 在工作站開啟一個server,將手機錄下的影像傳到工作站解碼後再傳回手機。
    \item 使用PWM,將LED調整到適合的亮度。
\end{enumerate}

\section*{工作分配}
\begin{itemize}[leftmargin=!,itemindent=-4em]
    \item 吳哲宇:編寫Zigduino傳輸程式,以及實作影像解碼程式。
    \item 李紹詮:修改Google提供的camera2 API範例App,手動設定曝光以及將錄好的影片自動上傳到工作站進行解碼。
    % \item 王欣維:
    \item 邱劭同:尋找影像辨識相關的研究,試圖減少辨識條紋所需的時間。
    \item 蔡存哲:尋找影像辨識相關的研究,試圖減少辨識條紋所需的時間。
\end{itemize}

\end{document}
