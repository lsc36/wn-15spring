\documentclass[10pt,a4paper]{article}
\usepackage[margin=1in]{geometry}
\usepackage{xeCJK}
%\usepackage{indentfirst}
\usepackage{enumitem}
\usepackage{setspace}
\usepackage{float}
\usepackage{calc}
\usepackage{amsmath}
\setCJKmainfont{AR PL UMing TW}

\usepackage{fancyhdr}
\pagestyle{fancy}
\lhead{WNFA Lab 4 Report}
\chead{}
\rhead{Group 5}
\lfoot{}
\cfoot{}
\rfoot{\thepage}
\renewcommand{\headrulewidth}{0.4pt}
\renewcommand{\footrulewidth}{0.4pt}

\setlength{\parindent}{0pt}
\setlength{\parskip}{.4em}
\linespread{1.15}

\begin{document}
\thispagestyle{fancy}

\begin{center}
    \LARGE WNFA Lab 4 Report
\end{center}
\begin{center}\begin{tabular}{lccr}
    Group 5: & B02902125吳哲宇 & B01902010李紹詮 & B01902048王欣維 \\
    & B01902085邱劭同 & B01902138蔡存哲
\end{tabular}\end{center}

\section*{工作原理}
\subsection*{MAC Protocol}
信號發送器擁有一個狀態變數(在程式中的\texttt{tx.state}),當狀態為0時開始DIFS等待,並將狀態設為1。當狀態為1時判斷通道狀態,通道不可用時重新開始DIFS等待,若通道可用且DIFS已完成則開始等待一段backoff(固定的\texttt{BACKOFF}加上一個隨機值)。在backoff等待結束後若通道依然可用,則發送器會開始發送封包。

\subsection*{Routing Algorithm}
我們的路由採取廣播的機制,每一個收到ping封包的節點會檢查自己是否已經收過此封包,若是沒有則會將封包再次廣播出去,若已經收過就不再次廣播。再次廣播時會把自己的ID加入封包中的路徑表,藉此得知傳輸經過的路徑。回傳ping reply時直接沿著request的路徑單播回去。

選擇這一個機制的原因是這個機制非常直觀,容易實作,而且就算其中一個節點被移除也不會造成太大的影響。缺點是channel會很busy,latency相對較高。

\section*{遇到的問題 \& 解決方法}
\begin{enumerate}
    \item 在Serial印debug message很花時間,會影響protocol正常運作\\
    Solution: 盡量不印message、提高baud rate
    \item Zigduino本身時脈不高,用迴圈在做carrier sensing的時候取樣率不夠高\\
    Solution: DIFS需要設在毫秒等級
\end{enumerate}

\section*{工作分配}
\begin{itemize}[leftmargin=!,itemindent=-4em]
    \item 吳哲宇: MAC protocol實作
    \item 李紹詮: broadcast ping實作
    \item 邱劭同: 測試、報告統整
    \item 蔡存哲: 測試
\end{itemize}

\end{document}
